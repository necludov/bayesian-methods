% -===============                                    ===============%
% -===============                                    ===============%
% -===============                                    ===============%
%                           PACKAGES                                 %
% -===============                                    ===============%
% -===============                                    ===============%
% -===============                                    ===============%

\usepackage{silence}
\WarningFilter*{hyperref}{Token not allowed in a PDF string}
\WarningsOff[caption]
\usepackage[utf8]{inputenc}
\usepackage[T1]{fontenc}
\usepackage{tabularx}
\usepackage{hyperref}
\usepackage{lipsum}
\usepackage{url}
\usepackage{placeins}
\usepackage{booktabs}
\usepackage{soul}
\usepackage{amsfonts}
\usepackage{nicefrac}
\usepackage{microtype}
\usepackage{natbib}
\usepackage{proba}
\usepackage{wrapfig}
\usepackage{caption}
\usepackage{amssymb}
\usepackage{pifont}
\newcommand{\cmark}{\ding{51}}
\newcommand{\xmark}{\ding{55}}
\usepackage{subcaption}
\usepackage{float}
\usepackage{rotating}
\usepackage{etoolbox}
\usepackage{xparse}
\usepackage{grffile}
\usepackage{amsmath}
\usepackage{xspace}
\usepackage{euscript}
\usepackage{enumitem}
\usepackage{mathtools}
\usepackage{tikz}
\usepackage{tablefootnote}
\usepackage[flushleft]{threeparttable}
\usepackage{multirow}
\usepackage[title]{appendix}
\usepackage{arydshln}
\usepackage{array, booktabs}
\usepackage{comment}
\usepackage{graphicx}
\usepackage{adjustbox}
\usepackage{dsfont}
\usepackage{amsmath,amsthm}
\usepackage{balance}
\usepackage{multicol}
\usepackage{xcolor}
\usepackage{nameref}
\usepackage{pdfpages}
\usepackage{braket}
\usepackage[normalem]{ulem}
\usepackage{bbding}
\usepackage[capitalise,noabbrev]{cleveref}
\usepackage{tcolorbox}
\usepackage{xfrac}
\usepackage[usestackEOL]{stackengine}
\usepackage{indentfirst}
\usepackage{csquotes}
\usepackage{pgf}
\usepackage{varwidth}
\usepackage{verbatimbox}
\usepackage{verbatim}
\usepackage{bm}
\usepackage[colorinlistoftodos,prependcaption,textsize=small]{todonotes}

\usepackage{empheq}
\definecolor{myblue}{rgb}{.8, .8, 1}
\newcommand*\mybluebox[1]{%
\colorbox{myblue}{\hspace{1em}#1\hspace{1em}}}

% -===============                                    ===============%
% -===============                                    ===============%
% -===============                                    ===============%
%                           CAPTION SETTINGS                         %
% -===============                                    ===============%
% -===============                                    ===============%
% -===============                                    ===============%

\captionsetup{font=small}
\captionsetup[sub]{font=small}
\captionsetup[subfigure]{labelformat = parens, labelsep = space, font = small}
\newcommand{\figleft}{{\em (Left)}}
\newcommand{\figcenter}{{\em (Center)}}
\newcommand{\figright}{{\em (Right)}}
\newcommand{\figtop}{{\em (Top)}}
\newcommand{\figbottom}{{\em (Bottom)}}
\newcommand{\captiona}{{\em (a)}}
\newcommand{\captionb}{{\em (b)}}
\newcommand{\captionc}{{\em (c)}}
\newcommand{\captiond}{{\em (d)}}

% -===============                                    ===============%
% -===============                                    ===============%
% -===============                                    ===============%
%                         COLOR MACROS                               %
% -===============                                    ===============%
% -===============                                    ===============%
% -===============                                    ===============%

\definecolor{pastelblue}{RGB}{76,113,175}
\definecolor{pastelgreen}{RGB}{144,238,144}
\definecolor{pastelred}{RGB}{196,78,82}
\definecolor{pastelgrey}{RGB}{230,230,230}
\definecolor{pastelbeige}{RGB}{243,236,221}
\definecolor{pastelpurple}{RGB}{154,139,192}
\definecolor{salmon}{RGB}{250, 128, 114}
\definecolor{darkgreen}{rgb}{0,0.6,0}
\definecolor{darkred}{rgb}{0.5,0,0}
\definecolor{verylightgreen}{HTML}{F6FFF9}
\definecolor{verylightred}{HTML}{FFF4F3}
\definecolor{verylightgray}{HTML}{F4F6F6}
\definecolor{babyblueeyes}{rgb}{0.63, 0.79, 0.95}
\definecolor{lightpink}{rgb}{1.00, 0.714, 0.757}


% -===============                                    ===============%
% -===============                                    ===============%
% -===============                                    ===============%
%                           TIKZ MACROS                              %
% -===============                                    ===============%
% -===============                                    ===============%
% -===============                                    ===============%
\usepackage{tikzpeople}
\usetikzlibrary{shapes,decorations,arrows,calc,arrows.meta,fit,positioning}

\tikzset{
    -Latex,auto,node distance =1 cm and 1 cm,semithick,
    state/.style ={ellipse, draw, minimum width = 0.7 cm},
    point/.style = {circle, draw, inner sep=0.04cm,fill,node contents={}},
    bidirected/.style={Latex-Latex,dashed},
    el/.style = {inner sep=2pt, align=left, sloped}
}

% Warning symbol
\newcommand\Warning{%
 \makebox[1.4em][c]{%
 \makebox[0pt][c]{\raisebox{.1em}{\small!}}%
 \makebox[0pt][c]{\color{red}\Large$\bigtriangleup$}}}%

% Text box

\newcommand{\centertikztextbox}[2]{{
\begin{tikzpicture}
\node[draw, align=center,fill=#1,thick](3) at (0,0) { 
\begin{varwidth}{0.9\linewidth}
#2
\end{varwidth}
};
\end{tikzpicture}
}
}

\newcommand{\tikztextbox}[2]{{
\begin{flushleft}
\begin{tikzpicture}
\node[draw, align=left,fill=#1,thick](3) at (0,0) { 
\begin{varwidth}{0.9\linewidth}
#2
\end{varwidth}
};
\end{tikzpicture}
\end{flushleft}
}
}
% Dashed version
\newcommand{\dashedtikztextbox}[3]{{
\begin{flushleft}
\begin{tikzpicture}
\hspace{#2}
\node[draw, align=left,fill=#1, thick, dashed](3) at (0,0) { 
\begin{varwidth}{0.9\linewidth}
#3
\end{varwidth}
};
\end{tikzpicture}
\end{flushleft}
}
}

% Arrow: [ style arguments, eat up, eat down, move right ]
\newcommand{\tikzarrow}[4]{{
\vspace{-#2}
\begin{flushleft}
\begin{tikzpicture}
\hspace{#4}
\node[align=left](1) at (0,0) { 
\begin{varwidth}{0.9\linewidth}
\end{varwidth}
};
\node[align=left](2) at (0,-1.5) { 
\begin{varwidth}{0.9\linewidth}
\end{varwidth}
};
\path[#1,thick] (1) edge (2);
\end{tikzpicture}
\end{flushleft}
\vspace{-#3}
}
}

%Green check
\newcommand{\greencheck}{}%
\DeclareRobustCommand{\greencheck}{%
  \tikz\fill[scale=0.5, color=pastelgreen]
  (0,.35) -- (.25,0) -- (1,.7) -- (.25,.15) -- cycle;
  
 
% This is how you draw a graph in text:
\usetikzlibrary{shapes.misc}
\newcommand*{\graphdraw}{{\scriptsize
		\tikz[scale=0.1]{
			%%% edges
			\draw[thin] (2,0) -- (0,0) -- (0,2) -- (2,2);
			%% vertices
			\draw[fill=blue,draw=blue] (0,0) circle (12pt);
			\draw[fill=red,draw=red] (0,2) circle (12pt);
			\draw[fill=darkgreen,draw=darkgreen] (2,2) circle (12pt);
			\draw[fill=blue,draw=blue] (2,0) circle (12pt);
		}
}}}


% -===============                                    ===============%
% -===============                                    ===============%
% -===============                                    ===============%
%                         THEOREM ENVIRONMENTS                       %
% -===============                                    ===============%
% -===============                                    ===============%
% -===============                                    ===============%

\newtheorem{theorem}{Theorem}
\newtheorem{assumption}{Assumption}
\newtheorem{axiom}{Axiom}
\newtheorem{case}{Case}
\newtheorem{claim}{Claim}
\newtheorem{conclusion}{Conclusion}
\newtheorem{condition}{Condition}
\newtheorem{conjecture}{Conjecture}
\newtheorem{corollary}{Corollary}
\newtheorem{criterion}{Criterion}
\newtheorem{definition}{Definition}
\newtheorem{exercise}{Exercise}
\newtheorem{example}{Example}
\newtheorem{lemma}{Lemma}
\newtheorem{notation}{Notation}
\newtheorem{problem}{Problem}
\newtheorem{proposition}{Proposition}
\newtheorem{property}{Property}
\newtheorem{remark}{Remark}
\newtheorem{solution}{Solution}
\newtheorem{summary}{Summary}
\newtheorem{motivation}{Motivation}
\newtheorem{query}{Query}
\newtheorem{observation}{Observation}
\newtheorem{question}{Question}
% Let cleveref and thmtools work together
\makeatletter
\def\thmt@refnamewithcomma #1#2#3,#4,#5\@nil{%
	\@xa\def\csname\thmt@envname #1utorefname\endcsname{#3}%
	\ifcsname #2refname\endcsname
	\csname #2refname\expandafter\endcsname\expandafter{\thmt@envname}{#3}{#4}%
	\fi}
\makeatother
\Crefname{conjecture}{Conjecture}{Conjectures}
\Crefname{definition}{Definition}{Definitions}
\Crefname{observation}{Observation}{Observations}
\Crefname{assumption}{Assumption}{Assumptions}
\Crefname{axiom}{Axiom}{Axioms}
\Crefname{case}{Case}{Cases}
\Crefname{claim}{Claim}{Claims}
\Crefname{conclusion}{Conclusion}{Conclusions}
\Crefname{condition}{Condition}{Conditions}
\Crefname{criterion}{Criterion}{Criteria}
\Crefname{exercise}{Exercise}{Exercises}
\Crefname{example}{Example}{Examples}
\Crefname{notation}{Notation}{Notations}
\Crefname{problem}{Problem}{Problems}
\Crefname{property}{Property}{Properties}
\Crefname{remark}{Remark}{Remarks}
\Crefname{solution}{Solution}{Solutions}
\Crefname{summary}{Summary}{Summaries}
\Crefname{motivation}{Motivation}{Motivations}
\Crefname{query}{Query}{Queries}
\Crefname{question}{Question}{Questions}
% -===============                                    ===============%
% -===============                                    ===============%
% -===============                                    ===============%
%                           MATH MACROS                              %
% -===============                                    ===============%
% -===============                                    ===============%
% -===============                                    ===============%
\newcommand{\mathcolorbox}[2]{\colorbox{#1}{$\displaystyle #2$}}
\newcommand{\one}{\mathds{1}}
\newcommand{\negquad}{\mkern-18mu}
\newcommand{\red}[1]{{\color{red} #1}}
\DeclareMathOperator*{\argmin}{argmin}
\DeclareMathOperator*{\argmax}{argmax}
\newcommand*\dbar[1]{\overline{\overline{\lower0.2ex\hbox{$#1$}}}}
\newcommand{\supp}{\text{supp\xspace}}
\DeclareMathOperator{\Tr}{Tr}
\newcommand{\dom}{\text{dom}}
\newcommand{\ran}{\text{ran}}
\newcommand{\setsize}[1]{\mid \! #1 \! \mid}
\def\th{^{\text{th}}}
%Stats macros:
\newcommand{\KL}{D_{\mathrm{KL}}}
\newcommand{\pr}{P}
\newcommand{\Exp}{\mathds{E}}
\newcommand{\eqdist}{\stackrel{d}{=}}
\newcommand{\eqas}{\stackrel{\text{a.s.}}{=}}
\newcommand{\standarderror}{\mathrm{SE}}
\newcommand{\indep}{\perp\!\!\!\perp}
%Causality macros
\newcommand{\doop}{\text{do}}
\newcommand{\Parents}{\text{Pa}}
\newcommand{\parents}{\text{pa}}
%Graph macros
\newcommand{\iso}{\cong}
\newcommand{\notiso}{\not cong}
\newcommand{\aut}[1]{\text{Aut}(#1)}
%Machine learning macros:
\newcommand{\lr}{\alpha}
\newcommand{\reg}{\lambda}
\newcommand{\lf}{\mathcal{L}}
\newcommand{\sigmoid}{\sigma}
\newcommand{\softplus}{\zeta}
\newcommand{\relu}{\text{ReLU}}
%Font macros
\def\cA{{\mathcal{A}}}
\def\cB{{\mathcal{B}}}
\def\cC{{\mathcal{C}}}
\def\cD{{\mathcal{D}}}
\def\cE{{\mathcal{E}}}
\def\cF{{\mathcal{F}}}
\def\cG{{\mathcal{G}}}
\def\cH{{\mathcal{H}}}
\def\cI{{\mathcal{I}}}
\def\cJ{{\mathcal{J}}}
\def\cK{{\mathcal{K}}}
\def\cL{{\mathcal{L}}}
\def\cM{{\mathcal{M}}}
\def\cN{{\mathcal{N}}}
\def\cO{{\mathcal{O}}}
\def\cP{{\mathcal{P}}}
\def\cQ{{\mathcal{Q}}}
\def\cR{{\mathcal{R}}}
\def\cS{{\mathcal{S}}}
\def\cT{{\mathcal{T}}}
\def\cU{{\mathcal{U}}}
\def\cV{{\mathcal{V}}}
\def\cW{{\mathcal{W}}}
\def\cX{{\mathcal{X}}}
\def\cY{{\mathcal{Y}}}
\def\cZ{{\mathcal{Z}}}
\def\bA{{\bm{A}}}
\def\bB{{\bm{B}}}
\def\bC{{\bm{C}}}
\def\bD{{\bm{D}}}
\def\bE{{\bm{E}}}
\def\bF{{\bm{F}}}
\def\bG{{\bm{G}}}
\def\bH{{\bm{H}}}
\def\bI{{\bm{I}}}
\def\bJ{{\bm{J}}}
\def\bK{{\bm{K}}}
\def\bL{{\bm{L}}}
\def\bM{{\bm{M}}}
\def\bN{{\bm{N}}}
\def\bO{{\bm{O}}}
\def\bP{{\bm{P}}}
\def\bQ{{\bm{Q}}}
\def\bR{{\bm{R}}}
\def\bS{{\bm{S}}}
\def\bT{{\bm{T}}}
\def\bU{{\bm{U}}}
\def\bV{{\bm{V}}}
\def\bW{{\bm{W}}}
\def\bX{{\bm{X}}}
\def\bY{{\bm{Y}}}
\def\bZ{{\bm{Z}}}
\def\ba{{\bm{a}}}
\def\bb{{\bm{b}}}
\def\bc{{\bm{c}}}
\def\bd{{\bm{d}}}
\def\be{{\bm{e}}}
\def\bf{{\bm{f}}}
\def\bg{{\bm{g}}}
\def\bh{{\bm{h}}}
\def\bi{{\bm{i}}}
\def\bj{{\bm{j}}}
\def\bk{{\bm{k}}}
\def\bl{{\bm{l}}}
\def\bbm{{\bm{m}}}
\def\bn{{\bm{n}}}
\def\bo{{\bm{o}}}
\def\bp{{\bm{p}}}
\def\bq{{\bm{q}}}
\def\br{{\bm{r}}}
\def\bs{{\bm{s}}}
\def\bt{{\bm{t}}}
\def\bu{{\bm{u}}}
\def\bv{{\bm{v}}}
\def\bw{{\bm{w}}}
\def\bx{{\bm{x}}}
\def\by{{\bm{y}}}
\def\bz{{\bm{z}}}
%For multiset notation
\newcommand*{\lms}{\{\mskip-5mu\{}
\newcommand*{\rms}{\}\mskip-5mu\}}


% -===============                                    ===============%
% -===============                                    ===============%
% -===============                                    ===============%
%                         MATH FONT COMMANDS                         %
% -===============                                    ===============%
% -===============                                    ===============%
% -===============                                    ===============%

\DeclareFontFamily{U}{BOONDOX-calo}{\skewchar\font=45 }
\DeclareFontShape{U}{BOONDOX-calo}{m}{n}{
  <-> s*[1.05] BOONDOX-r-calo}{}
\DeclareFontShape{U}{BOONDOX-calo}{b}{n}{
  <-> s*[1.05] BOONDOX-b-calo}{}
\DeclareMathAlphabet{\mathcalb}{U}{BOONDOX-calo}{m}{n}
\SetMathAlphabet{\mathcalb}{bold}{U}{BOONDOX-calo}{b}{n}
\DeclareMathAlphabet{\mathbcalb}{U}{BOONDOX-calo}{b}{n}
\newcommand{\sff}[1]{ \text{ {\sffamily \textbf{  {#1} }} } }
\newcommand{\ecal}[1]{ \EuScript{#1} }

%% Removes commands not allowed in PDF fields, such as line breaks
\pdfstringdefDisableCommands{%
  \def\\{}%
  \def\texttt#1{<#1>}%
}

% -===============                                    ===============%
% -===============                                    ===============%
% -===============                                    ===============%
%                    THIS PAPER MACROS AND COMMANDS                  %
% -===============                                    ===============%
% -===============                                    ===============%
% -===============                                    ===============%


\renewcommand{\paragraph}[1]{{\noindent \textbf{#1.}}}
\newcommand{\Appendix}{supplement\xspace}
\newcommand{\method}{SOD\xspace}
\newcommand{\kir}[1]{{\color{pastelblue}[Kirill:#1]}}
\newcommand{\citehere}[0]{{\color{red}
$^{\text{cite}}$}}
\newcommand{\x}[1]{x^{(#1)}}
\newcommand{\y}[1]{y^{(#1)}}
\newcommand{\mat}[0]{{\tt stack\_rows}}
\newcommand{\grad}[1]{{\tt update}({#1})}

\newcommand{\maybe}{\textcolor{gray}{\checkmark\kern-1.1ex\raisebox{.7ex}{\rotatebox[origin=c]{125}{--}}}}

\newcommand{\no}{\raisebox{-.3em}{\rlap{\raisebox{.3em}{\hspace{1.4em}\scriptsize}}\includegraphics[height=0.9em]{pics/no.pdf}}\xspace}

\newcommand{\yes}{\raisebox{-.3em}{\rlap{\raisebox{.3em}{\hspace{1.4em}\scriptsize}}\includegraphics[height=0.9em]{pics/yes.pdf}}\xspace}