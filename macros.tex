% \usepackage[capitalize]{cleveref}
\global\long\def\cv#1#2{\left\{  #1\,\middle|\,#2\right\}  }
\let\originalleft\left
\let\originalright\right
\renewcommand{\left}{\mathopen{}\mathclose\bgroup\originalleft}
\renewcommand{\right}{\aftergroup\egroup\originalright}
\global\long\def\X{\mathcal{X}}
\global\long\def\K{\mathcal{K}}
\global\long\def\U{\mathcal{U}}
\global\long\def\A{\mathcal{A}}
\global\long\def\T{\mathcal{T}}
\global\long\def\P{\mathcal{P}}
\global\long\def\L{\mathcal{L}}
\global\long\def\S{\mathcal{S}}
\newcommand{\om}{\omega}
\global\long\def\g{\nabla}
\global\long\def\fr#1#2{\frac{#1}{#2}}
\newcommand{\cd}{\cdot}
\newcommand{\diver}[1]{\g \cdot \left( #1 \right)}
\global\long\def\norm#1{\left\lVert #1\right\rVert }
\global\long\def\inner#1#2{\left\langle #1, #2\right\rangle}
\global\long\def\pa{\mathbf{\partial}}
\global\long\def\im{\implies}
\global\long\def\EE{\mathbb{E}}
\global\long\def\RR{\mathbb{R}}
\global\long\def\eval#1{\left.#1\right|}
\newcommand{\kilbo}{\textsc{kilbo}}
% \DeclareMathOperator*{\argmax}{arg\,max}
% \DeclareMathOperator*{\argmin}{arg\,min}
\newcommand{\pat}{\frac{\pa}{\pa t}}
\global\long\def\fr#1#2{\frac{#1}{#2}}
% \newcommand{\Exp}[2]{\mathbb{E}_{#1}\left[ #2\right]}

\definecolor{antiquefuchsia}{rgb}{0.57, 0.36, 0.51}
\definecolor{amethyst}{rgb}{0.6, 0.4, 0.8}
\newcommand{\gs}[1]{\textcolor{amethyst}{#1}}

\newcommand{\deriv}[2]{\frac{\partial #1}{\partial #2}}
\newcommand{\id}{{\mathrm{id}}}
\newcommand*\Let[6]{\State #1 $\gets$ #2}
\newcommand{\proj}{{\mathrm{proj}}}
\newcommand{\orb}{{\mathrm{orb}}}
\newcommand{\ord}{{\mathrm{ord}}}
\newcommand{\mean}{\mathbb{E}}
\newcommand{\var}{{\rm I\kern-.3em D}}
% \newcommand{\prob}{{\rm P}}
\renewcommand{\vec}[1]{\bm{#1}}
\newcommand{\svec}[1]{\mathbf{#1}}
\newcommand{\cond}{\,|\,}
\newcommand{\Normal}{\mathcal{N}}
\newcommand{\diag}{\mathrm{diag}}
\newcommand{\eps}{\varepsilon}
\newcommand{\ent}{\mathcal{H}}
\DeclareMathOperator*{\infimum}{inf}
\DeclareMathOperator*{\supremum}{sup}
% \newtheorem{theorem}{Theorem}
% \newtheorem*{theorem*}{Theorem}
% \newtheorem{lemma}{Lemma}
% \newtheorem{proposition}{Proposition}
% \newtheorem*{proposition*}{Proposition}
% \newtheorem{corollary}{Corollary}
% \newtheorem{example}{Example}
% \newtheorem*{example*}{Example}
\DeclareMathSymbol{\shortminus}{\mathbin}{AMSa}{"39}


\tcbuselibrary{theorems}

\tcbset{mybox/.style={colback=cyan!5,colframe=cyan!35!black, width =\textwidth, boxrule=1.0pt, top=5pt,bottom=5pt,left=2pt, right=2pt},
% left=2mm, right=2mm,fonttitle=\bfseries}, 
% fontupper=\small,
  before upper=\setlength{\parindent}
  {0em}\everypar{{\setbox0\lastbox}\everypar{}}
}
\newtcolorbox{mybox}[1][]{mybox,#1}

% % \newtcbtheorem[auto counter,number within=section,crefname={Definition}{Definitions}]{mydef}{Definition}%
% % {colback=cyan!5,colframe=cyan!35!black,fonttitle=\bfseries,
% % subtitle style={boxrule=0.4pt,colback=cyan!50!red!25!white},title=Box ~\thetcbcounter $\mid$ #2, label={#1}}{th}

% \newtcbtheorem[auto counter,number within=section,crefname={prop}{props}]{myprop}{Proposition}%
% {colback=green!5,colframe=green!35!black,fonttitle=\bfseries}{th}


% ------------------------------- [WIDEBAR -------------------------------
\makeatletter
\let\save@mathaccent\mathaccent
\newcommand*\if@single[3]{%
  \setbox0\hbox{${\mathaccent"0362{#1}}^H$}%
  \setbox2\hbox{${\mathaccent"0362{\kern0pt#1}}^H$}%
  \ifdim\ht0=\ht2 #3\else #2\fi
  }
%The bar will be moved to the right by a half of \macc@kerna, which is computed by amsmath:
\newcommand*\rel@kern[1]{\kern#1\dimexpr\macc@kerna}
%If there's a superscript following the bar, then no negative kern may follow the bar;
%an additional {} makes sure that the superscript is high enough in this case:
\newcommand*\widebar[1]{\@ifnextchar^{{\wide@bar{#1}{0}}}{\wide@bar{#1}{1}}}
%Use a separate algorithm for single symbols:
\newcommand*\wide@bar[2]{\if@single{#1}{\wide@bar@{#1}{#2}{1}}{\wide@bar@{#1}{#2}{2}}}
\newcommand*\wide@bar@[3]{%
  \begingroup
  \def\mathaccent##1##2{%
%Enable nesting of accents:
    \let\mathaccent\save@mathaccent
%If there's more than a single symbol, use the first character instead (see below):
    \if#32 \let\macc@nucleus\first@char \fi
%Determine the italic correction:
    \setbox\z@\hbox{$\macc@style{\macc@nucleus}_{}$}%
    \setbox\tw@\hbox{$\macc@style{\macc@nucleus}{}_{}$}%
    \dimen@\wd\tw@
    \advance\dimen@-\wd\z@
%Now \dimen@ is the italic correction of the symbol.
    \divide\dimen@ 3
    \@tempdima\wd\tw@
    \advance\@tempdima-\scriptspace
%Now \@tempdima is the width of the symbol.
    \divide\@tempdima 10
    \advance\dimen@-\@tempdima
%Now \dimen@ = (italic correction / 3) - (Breite / 10)
    \ifdim\dimen@>\z@ \dimen@0pt\fi
%The bar will be shortened in the case \dimen@<0 !
    \rel@kern{0.6}\kern-\dimen@
    \if#31
      \overline{\rel@kern{-0.6}\kern\dimen@\macc@nucleus\rel@kern{0.4}\kern\dimen@}%
      \advance\dimen@0.4\dimexpr\macc@kerna
%Place the combined final kern (-\dimen@) if it is >0 or if a superscript follows:
      \let\final@kern#2%
      \ifdim\dimen@<\z@ \let\final@kern1\fi
      \if\final@kern1 \kern-\dimen@\fi
    \else
      \overline{\rel@kern{-0.6}\kern\dimen@#1}%
    \fi
  }%
  \macc@depth\@ne
  \let\math@bgroup\@empty \let\math@egroup\macc@set@skewchar
  \mathsurround\z@ \frozen@everymath{\mathgroup\macc@group\relax}%
  \macc@set@skewchar\relax
  \let\mathaccentV\macc@nested@a
%The following initialises \macc@kerna and calls \mathaccent:
  \if#31
    \macc@nested@a\relax111{#1}%
  \else
%If the argument consists of more than one symbol, and if the first token is
%a letter, use that letter for the computations:
    \def\gobble@till@marker##1\endmarker{}%
    \futurelet\first@char\gobble@till@marker#1\endmarker
    \ifcat\noexpand\first@char A\else
      \def\first@char{}%
    \fi
    \macc@nested@a\relax111{\first@char}%
  \fi
  \endgroup
}
\makeatother
% ------------------------------- WIDEBAR] -------------------------------